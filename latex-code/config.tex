\documentclass[
german,
english,
paper = a4,
%toc=flat,
toc=chapterentrywithdots,
captions=tableabove,
listof=entryprefix,
listof=leveldown,
fontsize=12pt,
numbers=noenddot,
parskip,
headsepline = true,
footsepline = true,
%listof = totoc,
%bibliography = totoc,
%index = totoc,
plainfootsepline = true,
plainheadsepline = true,
footheight=25pt,
headheight=25pt,
usegeometry,
]{scrreprt}



% Laden der Pakete %
\usepackage[ngerman,english]{babel}
\usepackage[utf8]{inputenc}
\usepackage{lmodern}
\usepackage[T1]{fontenc}
\usepackage{geometry}
\geometry{
	includeheadfoot,
	top = 15mm,
	left = 20mm,
	right = 25mm, 
	bottom = 20mm,
	bindingoffset = 10mm,
	}

%\usepackage[letterspace=150]{microtype}
\usepackage[onehalfspacing]{setspace}
\BeforeStartingTOC[toc]{\singlespacing} 
\BeforeStartingTOC[lot]{\singlespacing\renewcommand\autodot{:}}
\BeforeStartingTOC[lof]{\singlespacing\renewcommand\autodot{:}}

%style=verbose-ibid,backend=bibtex
%\usepackage[style=verbose-ibid,backend=bibtex]{biblatex}
%\usepackage[backend=biber,style=numeric-comp,sorting=none]{biblatex}
%\addbibresource{Literatur.bib}
%custom footcite command for citation
\newcommand{\myfootcite}[2]{\footnote{\citeauthor{#1}, \citetitle{#1}, S. {#2} \cite{#1}}}

\newcommand{\myRef}[1]{\autoref{#1} \textit{\nameref{#1}} (S. \pageref{#1})}
\newcommand{\myfootref}[2]{\footnote{siehe \autoref{#1}, {#2}, S.\pageref{#1}}}
\newcommand{\customcode}[1]{\colorbox{codeBackground}{\rlap{\textbf{#1}}\hspace{\linewidth}\hspace{-2\fboxsep}}}
\usepackage{listings}
\usepackage{makeidx}
%\usepackage{csquotes}
%\usepackage{svg}

%\usepackage[bottom]{footmisc}
%\usepackage[acronym,nopostdot,nogroupskip,style=super,nonumberlist,section=section]{glossaries}
%\usepackage[acronym,nopostdot,nogroupskip,style=mcolalttree,nonumberlist]{glossaries}
%\usepackage[acronym,nopostdot,toc,style=super,nonumberlist,section=chapter]{glossaries}
%\makeglossaries
%\usepackage{scrlayer-scrpage}
%\clearscrheadfoot
%\ihead{\automark{}}
%\newenvironment{absolutelynopagebreak}
%  {\par\nobreak\vfil\penalty0\vfilneg
%   \vtop\bgroup}
%  {\par\xdef\tpd{\the\prevdepth}\egroup
%   \prevdepth=\tpd}




%#######################old pagestyle ###################################
%\cfoot{{--~}\pagemark{~--}}
%\ofoot{Maurice Götze}
%\ifoot{HTWK Leipzig}



%\ohead{\headmark}
%\automark{chapter}
%\renewcommand*{\chapterpagestyle}{scrheadings}
%\RedeclareSectionCommand[%
%  beforeskip=150pt,
%  afterskip=50pt
%]{chapter}
\RedeclareSectionCommands[%
  beforeskip=0ex,
  %afterskip=1pt plus .1\baselineskip minus .167\baselineskip
  afterskip=0.1ex
]{section,subsection,subsubsection}
%#######################old pagestyle ###################################



%#############################pagestyle with chapter title############################
\RedeclareSectionCommand[
  prefixfont = \Large,% \bfseries\sffamily ist über Schriftelement disposition% voreingestellt
        font = \Huge,
  beforeskip = -5ex,% minus, damit erster Abschnitt nicht eingezogen wird
   innerskip = 0cm,
   afterskip = 0ex
  ]{chapter}
 
%\renewcommand\raggedchapter{\raggedleft}% Ausrichtung Kapitelüberschrift samt Präfixzeile

%\renewcommand\chapterformat{\textls*[600]{\chapapp}\space\space\HUGE\space\space\thechapter}
%\newcommand\HUGE{\fontsize{100pt}{105pt}\selectfont}
%#############################pagestyle with chapter title############################



\usepackage{multicol} %zweispaltig
\usepackage{paralist} %Aufzählungen mit kurzen Abständen
\usepackage{enumitem}

\usepackage{graphicx} 
\usepackage{pdfpages}
\usepackage{csquotes}
%\usepackage{siunitx}
%\sisetup{
%    locale=DE,
%    per-mode=fraction,
%}
%Mathematische Audrücke
\usepackage{amsmath,amssymb,amstext}
%Einfügen von Grafiken
\usepackage{graphicx}
\usepackage{float}
\graphicspath{ {./} }
\usepackage{caption}
\captionsetup{belowskip=-20pt}
%Links im Inhaltsverzeichnis
%\usepackage[hidelinks]{hyperref}
\usepackage{wrapfig}

\usepackage{tabularx}
\usepackage{xurl}
\usepackage{fancyhdr}
\usepackage{verbatim}
\usepackage{subfigure}
\usepackage{makecell}

%Neuerstellen eines Pagecounters
\usepackage{xassoccnt}
\newcounter{realpage}
\DeclareAssociatedCounters{page}{realpage}
\AtBeginDocument{%
  \stepcounter{realpage}
}

%\usepackage{scrlayer-scrpage}                   % für Kopf- und Fußzeile
%\clearpairofpagestyles
\definecolor{gray}{rgb}{0.5,0.5,0.5}
\newcommand{\chaptercolor}{gray}
\usepackage[hypertexnames=true,hidelinks=true]{hyperref}
\definecolor{lst-gray}{rgb}{0.98,0.98,0.98}
\definecolor{lst-blue}{RGB}{40,0.0,255}
\definecolor{lst-green}{RGB}{65,128,95}
\definecolor{lst-red}{RGB}{200,0,85}
\definecolor{lst-variable}{RGB}{100,0,85}
\usepackage{imakeidx}
\usepackage{lastpage}
\usepackage{listings}
\renewcommand{\lstlistingname}{Quelltext}
\renewcommand{\lstlistlistingname}{Quelltextverzeichnis}
\lstdefinelanguage{PowerShell}{
	morekeywords={
		Add-Content,Add-PSSnapin,Clear-Content,Clear-History,Clear-Host,Clear-Item,Clear-ItemProperty,Clear-Variable,Compare-Object,Connect-PSSession,Connect-VIServer,Convert-ExcelRangeToImage,ConvertFrom-ExcelData,ConvertFrom-ExcelSheet,ConvertFrom-String,Convert-Path,Copy-Item,Copy-ItemProperty,Disable-PSBreakpoint,Disconnect-PSSession,Enable-PSBreakpoint,Enter-PSSession,Exit-PSSession,Export-Alias,Export-Csv,Export-PSSession,Export-VApp,ForEach-Object,Format-Custom,Format-Hex,Format-List,Format-Table,Format-Wide,Get-Alias,Get-ChildItem,Get-Clipboard,Get-Command,Get-ComputerInfo,Get-Content,Get-History,Get-Item,Get-ItemProperty,Get-ItemPropertyValue,Get-Job,Get-Location,Get-Member,Get-Module,Get-PowerCLIConfiguration,Get-PowerCLIHelp,Get-PowerCLIVersion,Get-Process,Get-PSBreakpoint,Get-PSCallStack,Get-PSDrive,Get-PSSession,Get-PSSnapin,Get-Service,Get-TimeZone,Get-Unique,Get-Variable,Get-WmiObject,Group-Object,help,Import-Alias,Import-Csv,Import-Module,Import-PSSession,Invoke-Command,Invoke-DrsRecommendation,Invoke-Expression,Invoke-History,Invoke-Item,Invoke-RestMethod,Invoke-VMHostProfile,Invoke-WebRequest,Invoke-WmiMethod,Measure-Object,mkdir,Move-Item,Move-ItemProperty,New-Alias,New-ExcelChartDefinition,New-Item,New-Module,New-PSDrive,New-PSSession,New-PSSessionConfigurationFile,New-Variable,Out-GridView,Out-Host,Out-Printer,Pop-Location,powershell_ise.exe,Push-Location,Receive-Job,Receive-PSSession,Remove-Item,Remove-ItemProperty,Remove-Job,Remove-Module,Remove-PSBreakpoint,Remove-PSDrive,Remove-PSSession,Remove-PSSnapin,Remove-Variable,Remove-WmiObject,Rename-Item,Rename-ItemProperty,Resolve-Path,Resume-Job,Select-Object,Select-String,Set-Alias,Set-Clipboard,Set-Content,Set-ExcelColumn,Set-ExcelRange,Set-ExcelRow,Set-Item,Set-ItemProperty,Set-Location,Set-PowerCLIConfiguration,Set-PSBreakpoint,Set-TimeZone,Set-Variable,Set-VMQuestion,Set-WmiInstance,Show-Command,Sort-Object,Start-Job,Start-Process,Start-Service,Start-Sleep,Stop-Job,Stop-Process,Stop-Service,Stop-VMGuest,Suspend-Job,Tee-Object,Trace-Command,Wait-Job,Where-Object,Write-Output
	},
	morekeywords={
		Add-AppvClientConnectionGroup,Add-AppvClientPackage,Add-AppvPublishingServer,Add-AppxPackage,Add-AppxProvisionedPackage,Add-AppxVolume,Add-BitsFile,Add-CertificateEnrollmentPolicyServer,Add-Computer,Add-Content,Add-History,Add-JobTrigger,Add-KdsRootKey,Add-LocalGroupMember,Add-Member,Add-PassthroughDevice,Add-PSSnapin,Add-SignerRule,Add-Type,Add-VirtualSwitchPhysicalNetworkAdapter,Add-VMHost,Add-VMHostNtpServer,Add-WindowsCapability,Add-WindowsDriver,Add-WindowsImage,Add-WindowsPackage,Checkpoint-Computer,Clear-AIPAuthentication,Clear-Content,Clear-EventLog,Clear-History,Clear-Item,Clear-ItemProperty,Clear-KdsCache,Clear-RecycleBin,Clear-Tpm,Clear-UevAppxPackage,Clear-UevConfiguration,Clear-Variable,Clear-WindowsCorruptMountPoint,Compare-Object,Complete-BitsTransfer,Complete-DtiagnosticTransaction,Complete-Transaction,Confirm-SecureBootUEFI,Connect-CisServer,Connect-PSSession,Connect-VIServer,Connect-WSMan,ConvertFrom-CIPolicy,ConvertFrom-Csv,ConvertFrom-Json,ConvertFrom-SecureString,ConvertFrom-String,ConvertFrom-StringData,Convert-Path,Convert-String,ConvertTo-Csv,ConvertTo-Html,ConvertTo-Json,ConvertTo-ProcessMitigationPolicy,ConvertTo-SecureString,ConvertTo-TpmOwnerAuth,ConvertTo-Xml,Copy-ContentLibraryItem,Copy-DatastoreItem,Copy-HardDisk,Copy-Item,Copy-ItemProperty,Copy-VMGuestFile,Debug-Job,Debug-Process,Debug-Runspace,Delete-DeliveryOptimizationCache,Disable-AppBackgroundTaskDiagnosticLog,Disable-Appv,Disable-AppvClientConnectionGroup,Disable-ComputerRestore,Disable-JobTrigger,Disable-LocalUser,Disable-PSBreakpoint,Disable-PSRemoting,Disable-PSSessionConfiguration,Disable-RunspaceDebug,Disable-ScheduledJob,Disable-TlsCipherSuite,Disable-TlsEccCurve,Disable-TlsSessionTicketKey,Disable-TpmAutoProvisioning,Disable-Uev,Disable-UevAppxPackage,Disable-UevTemplate,Disable-WindowsErrorReporting,Disable-WindowsOptionalFeature,Disable-WSManCredSSP,Disconnect-CisServer,Disconnect-PSSession,Disconnect-VIServer,Disconnect-WSMan,Dismount-AppxVolume,Dismount-Tools,Dismount-WindowsImage,Edit-CIPolicyRule,Enable-AppBackgroundTaskDiagnosticLog,Enable-Appv,Enable-AppvClientConnectionGroup,Enable-ComputerRestore,Enable-JobTrigger,Enable-LocalUser,Enable-PSBreakpoint,Enable-PSRemoting,Enable-PSSessionConfiguration,Enable-RunspaceDebug,Enable-ScheduledJob,Enable-TlsCipherSuite,Enable-TlsEccCurve,Enable-TlsSessionTicketKey,Enable-TpmAutoProvisioning,Enable-Uev,Enable-UevAppxPackage,Enable-UevTemplate,Enable-WindowsErrorReporting,Enable-WindowsOptionalFeature,Enable-WSManCredSSP,Enter-PSHostProcess,Enter-PSSession,Exit-PSHostProcess,Exit-PSSession,Expand-WindowsCustomDataImage,Expand-WindowsImage,Export-AIPLogs,Export-Alias,Export-BinaryMiLog,Export-Certificate,Export-Clixml,Export-Console,Export-ContentLibraryItem,Export-Counter,Export-Csv,Export-FormatData,Export-LcmClusterDesiredState,Export-ModuleMember,Export-PfxCertificate,Export-ProvisioningPackage,Export-PSSession,Export-StartLayout,Export-StartLayoutEdgeAssets,Export-TlsSessionTicketKey,Export-Trace,Export-UevConfiguration,Export-UevPackage,Export-VApp,Export-VMHostProfile,Export-WindowsCapabilitySource,Export-WindowsDriver,Export-WindowsImage,Find-Package,Find-PackageProvider,ForEach-Object,Format-Custom,Format-List,Format-SecureBootUEFI,Format-Table,Format-VMHostDiskPartition,Format-Wide,Get-Acl,Get-AdvancedSetting,Get-AIPFileStatus,Get-AIPScannerConfiguration,Get-AIPScannerStatus,Get-AlarmAction,Get-AlarmActionTrigger,Get-AlarmDefinition,Get-AlarmTrigger,Get-Alias,Get-Annotation,Get-AppLockerFileInformation,Get-AppLockerPolicy,Get-AppvClientApplication,Get-AppvClientConfiguration,Get-AppvClientConnectionGroup,Get-AppvClientMode,Get-AppvClientPackage,Get-AppvPublishingServer,Get-AppvStatus,Get-AppxDefaultVolume,Get-AppxPackage,Get-AppxPackageManifest,Get-AppxProvisionedPackage,Get-AppxVolume,Get-AuthenticodeSignature,Get-BitsTransfer,Get-Drive,Get-Certificate,Get-CertificateAutoEnrollmentPolicy,Get-CertificateEnrollmentPolicyServer,Get-CertificateNotificationTask,Get-ChildItem,Get-CimAssociatedInstance,Get-CimClass,Get-CimInstance,Get-CimSession,Get-CIPolicy,Get-CIPolicyIdInfo,Get-CIPolicyInfo,Get-CisService,Get-Clipboard,Get-Cluster,Get-CmsMessage,Get-Command,Get-ComputerInfo,Get-ComputerRestorePoint,Get-Content,Get-ContentLibrary,Get-ContentLibraryItem,Get-ControlPanelItem,Get-Counter,Get-Credential,Get-Culture,Get-CustomAttribute,Get-DAPolicyChange,Get-Datacenter,Get-Datastore,Get-DatastoreCluster,Get-Date,Get-DeliveryOptimizationLog,Get-DeliveryOptimizationPerfSnap,Get-DeliveryOptimizationPerfSnapThisMonth,Get-DeliveryOptimizationStatus,Get-DOConfig,Get-DODownloadMode,Get-DOPercentageMaxBackgroundBandwidth,Get-DOPercentageMaxForegroundBandwidth,Get-DrsClusterGroup,Get-DrsRecommendation,Get-DrsRule,Get-DrsVMHostRule,Get-ErrorReport,Get-EsxCli,Get-EsxTop,Get-Event,Get-EventLog,Get-EventSubscriber,Get-EventType,Get-ExecutionPolicy,Get-FloppyDrive,Get-Folder,Get-FormatData,Get-HAPrimaryVMHost,Get-HardDisk,Get-Help,Get-History,Get-Host,Get-HotFix,Get-Inventory,Get-IScsiHbaTarget,Get-Item,Get-ItemProperty,Get-ItemPropertyValue,Get-Job,Get-JobTrigger,Get-KdsConfiguration,Get-KdsRootKey,Get-LcmClusterDesiredStateRecommendation,Get-LcmHardwareCompatibility,Get-LcmImage,Get-LocalGroup,Get-LocalGroupMember,Get-LocalUser,Get-Location,Get-Log,Get-LogType,Get-Member,Get-Metric,Get-MIPNetworkDiscoveryConfiguration,Get-MIPNetworkDiscoveryJobs,Get-MIPNetworkDiscoveryStatus,Get-Module,Get-NetworkAdapter,Get-NicTeamingPolicy,Get-NonRemovableAppsPolicy,Get-OSCustomizationNicMapping,Get-OSCustomizationSpec,Get-OvfConfiguration,Get-Package,Get-PackageProvider,Get-PackageSource,Get-PassthroughDevice,Get-PfxCertificate,Get-PfxData,Get-PmemDisk,Get-PmemPhysicalDevice,Get-PmemUnusedRegion,Get-PowerCLIConfiguration,Get-PowerCLIContext,Get-PowerCLIVersion,Get-Process,Get-ProcessMitigation,Get-ProvisioningPackage,Get-PSBreakpoint,Get-PSCallStack,Get-PSDrive,Get-PSHostProcessInfo,Get-PSProvider,Get-PSReadLineKeyHandler,Get-PSReadLineOption,Get-PSSession,Get-PSSessionCapability,Get-PSSessionConfiguration,Get-PSSnapin,Get-Random,Get-ResourcePool,Get-Runspace,Get-RunspaceDebug,Get-ScheduledJob,Get-ScheduledJobOption,Get-ScsiController,Get-ScsiLun,Get-ScsiLunPath,Get-SecureBootPolicy,Get-SecureBootUEFI,Get-SecurityPolicy,Get-Service,Get-Snapshot,Get-Stat,Get-StatInterval,Get-StatType,Get-SystemDriver,Get-Tag,Get-TagAssignment,Get-TagCategory,Get-Task,Get-Template,Get-TimeZone,Get-TlsCipherSuite,Get-TlsEccCurve,Get-Tpm,Get-TpmEndorsementKeyInfo,Get-TpmSupportedFeature,Get-TraceSource,Get-Transaction,Get-TroubleshootingPack,Get-TrustedProvisioningCertificate,Get-TypeData,Get-UevAppxPackage,Get-UevConfiguration,Get-UevStatus,Get-UevTemplate,Get-UevTemplateProgram,Get-UICulture,Get-Unique,Get-UsbDevice,Get-VApp,Get-Variable,Get-VIAccount,Get-VICredentialStoreItem,Get-VIEvent,Get-View,Get-VIObjectByVIView,Get-VIPermission,Get-VIPrivilege,Get-VIProperty,Get-VIRole,Get-VirtualNetwork,Get-VirtualPortGroup,Get-VirtualSwitch,Get-VM,Get-VMGuest,Get-VMGuestDisk,Get-VMHost,Get-VMHostAccount,Get-VMHostAdvancedConfiguration,Get-VMHostAuthentication,Get-VMHostAvailableTimeZone,Get-VMHostDiagnosticPartition,Get-VMHostDisk,Get-VMHostDiskPartition,Get-VMHostFirewallDefaultPolicy,Get-VMHostFirewallException,Get-VMHostFirmware,Get-VMHostHardware,Get-VMHostHba,Get-VMHostModule,Get-VMHostNetwork,Get-VMHostNetworkAdapter,Get-VMHostNetworkStack,Get-VMHostNtpServer,Get-VMHostPatch,Get-VMHostPciDevice,Get-VMHostProfile,Get-VMHostProfileImageCacheConfiguration,Get-VMHostProfileRequiredInput,Get-VMHostProfileStorageDeviceConfiguration,Get-VMHostProfileUserConfiguration,Get-VMHostProfileVmPortGroupConfiguration,Get-VMHostRoute,Get-VMHostService,Get-VMHostSnmp,Get-VMHostStartPolicy,Get-VMHostStorage,Get-VMHostSysLogServer,Get-VMQuestion,Get-VMResourceConfiguration,Get-VMStartPolicy,Get-WheaMemoryPolicy,Get-WIMBootEntry,Get-WinAcceptLanguageFromLanguageListOptOut,Get-WinCultureFromLanguageListOptOut,Get-WinDefaultInputMethodOverride,Get-WindowsCapability,Get-WindowsDeveloperLicense,Get-WindowsDriver,Get-WindowsEdition,Get-WindowsErrorReporting,Get-WindowsImage,Get-WindowsImageContent,Get-WindowsOptionalFeature,Get-WindowsPackage,Get-WindowsSearchSetting,Get-WinEvent,Get-WinHomeLocation,Get-WinLanguageBarOption,Get-WinSystemLocale,Get-WinUILanguageOverride,Get-WinUserLanguageList,Get-WmiObject,Get-WSManCredSSP,Get-WSManInstance,Group-Object,Import-AIPScannerConfiguration,Import-Alias,Import-BinaryMiLog,Import-Certificate,Import-Clixml,Import-Counter,Import-Csv,Import-LcmClusterDesiredState,Import-LocalizedData,Import-MIPNetworkDiscoveryConfiguration,Import-Module,Import-PackageProvider,Import-PfxCertificate,Import-PSSession,Import-StartLayout,Import-TpmOwnerAuth,Import-UevConfiguration,Import-VApp,Import-VMHostProfile,Initialize-PmemPhysicalDevice,Initialize-Tpm,Install-AIPScanner,Install-MIPNetworkDiscovery,Install-Package,Install-PackageProvider,Install-ProvisioningPackage,Install-TrustedProvisioningCertificate,Install-VMHostPatch,Invoke-CimMethod,Invoke-Command,Invoke-CommandInDesktopPackage,Invoke-DrsRecommendation,Invoke-DscResource,Invoke-Expression,Invoke-History,Invoke-Item,Invoke-RestMethod,Invoke-TroubleshootingPack,Invoke-VMHostProfile,Invoke-VMScript,Invoke-WebRequest,Invoke-WmiMethod,Invoke-WSManAction,Join-DtiagnosticResourceManager,Join-Path,Limit-EventLog,Measure-Command,Measure-Object,Merge-CIPolicy,Mount-AppvClientConnectionGroup,Mount-AppvClientPackage,Mount-AppxVolume,Mount-Tools,Mount-WindowsImage,Move-AppxPackage,Move-Cluster,Move-Datacenter,Move-Datastore,Move-Folder,Move-HardDisk,Move-Inventory,Move-Item,Move-ItemProperty,Move-ResourcePool,Move-Template,Move-VApp,Move-VM,Move-VMHost,New-AdvancedSetting,New-AIPCustomPermissions,New-AlarmAction,New-AlarmActionTrigger,New-AlarmDefinition,New-AlarmTrigger,New-Alias,New-AppLockerPolicy,New-Drive,New-CertificateNotificationTask,New-CimInstance,New-CimSession,New-CimSessionOption,New-CIPolicy,New-CIPolicyRule,New-Cluster,New-ContentLibrary,New-ContentLibraryItem,New-CustomAttribute,New-Datacenter,New-Datastore,New-DatastoreCluster,New-DrsClusterGroup,New-DrsRule,New-DrsVMHostRule,New-DtiagnosticTransaction,New-Event,New-EventLog,New-FileCatalog,New-FloppyDrive,New-Folder,New-HardDisk,New-IScsiHbaTarget,New-Item,New-ItemProperty,New-JobTrigger,New-LocalGroup,New-LocalUser,New-Module,New-ModuleManifest,New-NetIPsecAuthProposal,New-NetIPsecMainModeCryptoProposal,New-NetIPsecQuickModeCryptoProposal,New-NetworkAdapter,New-OAuthSecurityContext,New-Object,New-OSCustomizationNicMapping,New-OSCustomizationSpec,New-PmemDisk,New-ProvisioningRepro,New-PSDrive,New-PSRoleCapabilityFile,New-PSSession,New-PSSessionConfigurationFile,New-PSSessionOption,New-PSTransportOption,New-PSWorkflowExecutionOption,New-ResourcePool,New-ScheduledJobOption,New-ScsiController,New-SelfSignedCertificate,New-Service,New-Snapshot,New-StatInterval,New-Tag,New-TagAssignment,New-TagCategory,New-Template,New-TimeSpan,New-TlsSessionTicketKey,New-VApp,New-Variable,New-VICredentialStoreItem,New-VIPermission,New-VIProperty,New-VIRole,New-VirtualPortGroup,New-VirtualSwitch,New-VISamlSecurityContext,New-VM,New-VMHostAccount,New-VMHostNetworkAdapter,New-VMHostProfile,New-VMHostProfileVmPortGroupConfiguration,New-VMHostRoute,New-WebServiceProxy,New-WindowsCustomImage,New-WindowsImage,New-WinEvent,New-WinUserLanguageList,New-WSManInstance,New-WSManSessionOption,Open-VMConsoleWindow,Optimize-AppxProvisionedPackages,Optimize-WindowsImage,Out-Default,Out-File,Out-GridView,Out-Host,Out-Null,Out-Printer,Out-String,Pop-Location,Protect-CmsMessage,Publish-AppvClientPackage,Publish-DscConfiguration,Push-Location,Read-Host,Receive-DtiagnosticTransaction,Receive-Job,Receive-PSSession,Register-ArgumentCompleter,Register-CimIndicationEvent,Register-EngineEvent,Register-ObjectEvent,Register-PackageSource,Register-PSSessionConfiguration,Register-ScheduledJob,Register-UevTemplate,Register-WmiEvent,Remove-AdvancedSetting,Remove-AlarmAction,Remove-AlarmActionTrigger,Remove-AlarmDefinition,Remove-AppvClientConnectionGroup,Remove-AppvClientPackage,Remove-AppvPublishingServer,Remove-AppxPackage,Remove-AppxProvisionedPackage,Remove-AppxVolume,Remove-BitsTransfer,Remove-Drive,Remove-CertificateEnrollmentPolicyServer,Remove-CertificateNotificationTask,Remove-CimInstance,Remove-CimSession,Remove-CIPolicyRule,Remove-Cluster,Remove-Computer,Remove-ContentLibrary,Remove-ContentLibraryItem,Remove-CustomAttribute,Remove-Datacenter,Remove-Datastore,Remove-DatastoreCluster,Remove-DrsClusterGroup,Remove-DrsRule,Remove-DrsVMHostRule,Remove-Event,Remove-EventLog,Remove-FloppyDrive,Remove-Folder,Remove-HardDisk,Remove-Inventory,Remove-IScsiHbaTarget,Remove-Item,Remove-ItemProperty,Remove-Job,Remove-JobTrigger,Remove-LocalGroup,Remove-LocalGroupMember,Remove-LocalUser,Remove-Module,Remove-NetworkAdapter,Remove-OSCustomizationNicMapping,Remove-OSCustomizationSpec,Remove-PassthroughDevice,Remove-PmemDisk,Remove-PSBreakpoint,Remove-PSDrive,Remove-PSReadLineKeyHandler,Remove-PSSession,Remove-PSSnapin,Remove-ResourcePool,Remove-Snapshot,Remove-StatInterval,Remove-Tag,Remove-TagAssignment,Remove-TagCategory,Remove-Template,Remove-TypeData,Remove-UsbDevice,Remove-VApp,Remove-Variable,Remove-VICredentialStoreItem,Remove-VIPermission,Remove-VIProperty,Remove-VIRole,Remove-VirtualPortGroup,Remove-VirtualSwitch,Remove-VirtualSwitchPhysicalNetworkAdapter,Remove-VM,Remove-VMHost,Remove-VMHostAccount,Remove-VMHostNetworkAdapter,Remove-VMHostNtpServer,Remove-VMHostProfile,Remove-VMHostProfileVmPortGroupConfiguration,Remove-VMHostRoute,Remove-WindowsCapability,Remove-WindowsDriver,Remove-WindowsImage,Remove-WindowsPackage,Remove-WmiObject,Remove-WSManInstance,Rename-Computer,Rename-Item,Rename-ItemProperty,Rename-LocalGroup,Rename-LocalUser,Repair-AppvClientConnectionGroup,Repair-AppvClientPackage,Repair-UevTemplateIndex,Repair-WindowsImage,Reset-ComputerMachinePassword,Resolve-DnsName,Resolve-Path,Restart-Computer,Restart-Service,Restart-VM,Restart-VMGuest,Restart-VMHost,Restart-VMHostService,Restore-Computer,Restore-UevBackup,Restore-UevUserSetting,Resume-BitsTransfer,Resume-Job,Resume-ProvisioningSession,Resume-Service,Save-Help,Save-Package,Save-WindowsImage,Select-Object,Select-String,Select-Xml,Send-AppvClientReport,Send-DtiagnosticTransaction,Send-MailMessage,Set-Acl,Set-AdvancedSetting,Set-AIPAuthentication,Set-AIPFileClassification,Set-AIPFileLabel,Set-AIPScanner,Set-AIPScannerConfiguration,Set-AlarmDefinition,Set-Alias,Set-Annotation,Set-AppBackgroundTaskResourcePolicy,Set-AppLockerPolicy,Set-AppvClientConfiguration,Set-AppvClientMode,Set-AppvClientPackage,Set-AppvPublishingServer,Set-AppxDefaultVolume,Set-AppXProvisionedDataFile,Set-AuthenticodeSignature,Set-BitsTransfer,Set-Drive,Set-CertificateAutoEnrollmentPolicy,Set-CimInstance,Set-CIPolicyIdInfo,Set-CIPolicySetting,Set-CIPolicyVersion,Set-Clipboard,Set-Cluster,Set-Content,Set-ContentLibrary,Set-ContentLibraryItem,Set-Culture,Set-CustomAttribute,Set-Datacenter,Set-Datastore,Set-DatastoreCluster,Set-Date,Set-DeliveryOptimizationStatus,Set-DODownloadMode,Set-DOPercentageMaxBackgroundBandwidth,Set-DOPercentageMaxForegroundBandwidth,Set-DrsClusterGroup,Set-DrsRule,Set-DrsVMHostRule,Set-DscLocalConfigurationManager,Set-ExecutionPolicy,Set-FloppyDrive,Set-Folder,Set-HardDisk,Set-HVCIOptions,Set-IScsiHbaTarget,Set-Item,Set-ItemProperty,Set-JobTrigger,Set-KdsConfiguration,Set-LocalGroup,Set-LocalUser,Set-Location,Set-MIPNetworkDiscoveryConfiguration,Set-NetworkAdapter,Set-NicTeamingPolicy,Set-NonRemovableAppsPolicy,Set-OSCustomizationNicMapping,Set-OSCustomizationSpec,Set-PackageSource,Set-PowerCLIConfiguration,Set-ProcessMitigation,Set-PSBreakpoint,Set-PSDebug,Set-PSReadLineKeyHandler,Set-PSReadLineOption,Set-PSSessionConfiguration,Set-ResourcePool,Set-RuleOption,Set-ScheduledJob,Set-ScheduledJobOption,Set-ScsiController,Set-ScsiLun,Set-ScsiLunPath,Set-SecureBootUEFI,Set-SecurityPolicy,Set-Service,Set-Snapshot,Set-StatInterval,Set-StrictMode,Set-Tag,Set-TagCategory,Set-Template,Set-TimeZone,Set-TpmOwnerAuth,Set-TraceSource,Set-UevConfiguration,Set-UevTemplateProfile,Set-VApp,Set-Variable,Set-VIPermission,Set-VIRole,Set-VirtualPortGroup,Set-VirtualSwitch,Set-VM,Set-VMHost,Set-VMHostAccount,Set-VMHostAdvancedConfiguration,Set-VMHostAuthentication,Set-VMHostDiagnosticPartition,Set-VMHostFirewallDefaultPolicy,Set-VMHostFirewallException,Set-VMHostFirmware,Set-VMHostHba,Set-VMHostModule,Set-VMHostNetwork,Set-VMHostNetworkAdapter,Set-VMHostNetworkStack,Set-VMHostProfile,Set-VMHostProfileImageCacheConfiguration,Set-VMHostProfileStorageDeviceConfiguration,Set-VMHostProfileUserConfiguration,Set-VMHostProfileVmPortGroupConfiguration,Set-VMHostRoute,Set-VMHostService,Set-VMHostSnmp,Set-VMHostStartPolicy,Set-VMHostStorage,Set-VMHostSysLogServer,Set-VMQuestion,Set-VMResourceConfiguration,Set-VMStartPolicy,Set-WheaMemoryPolicy,Set-WinAcceptLanguageFromLanguageListOptOut,Set-WinCultureFromLanguageListOptOut,Set-WinDefaultInputMethodOverride,Set-WindowsEdition,Set-WindowsProductKey,Set-WindowsSearchSetting,Set-WinHomeLocation,Set-WinLanguageBarOption,Set-WinSystemLocale,Set-WinUILanguageOverride,Set-WinUserLanguageList,Set-WmiInstance,Set-WSManInstance,Set-WSManQuickConfig,Show-Command,Show-ControlPanelItem,Show-EventLog,Show-WindowsDeveloperLicenseRegistration,Sort-Object,Split-Path,Split-WindowsImage,Start-AIPScan,Start-AIPScannerDiagnostics,Start-BitsTransfer,Start-DscConfiguration,Start-DtiagnosticResourceManager,Start-Job,Start-MIPNetworkDiscovery,Start-OSUninstall,Start-Process,Start-Service,Start-Sleep,Start-Transaction,Start-Transcript,Start-VApp,Start-VM,Start-VMHost,Start-VMHostService,Stop-AIPScan,Stop-AppvClientConnectionGroup,Stop-AppvClientPackage,Stop-Computer,Stop-DtiagnosticResourceManager,Stop-Job,Stop-Process,Stop-Service,Stop-Task,Stop-Transcript,Stop-VApp,Stop-VM,Stop-VMGuest,Stop-VMHost,Stop-VMHostService,Suspend-BitsTransfer,Suspend-Job,Suspend-Service,Suspend-VM,Suspend-VMGuest,Suspend-VMHost,Switch-Certificate,Sync-AppvPublishingServer,Tee-Object,Test-AppLockerPolicy,Test-Certificate,Test-ComputerSecureChannel,Test-Connection,Test-DscConfiguration,Test-FileCatalog,Test-KdsRootKey,Test-LcmClusterCompliance,Test-LcmClusterHealth,Test-ModuleManifest,Test-Path,Test-PSSessionConfigurationFile,Test-UevTemplate,Test-VMHostProfileCompliance,Test-VMHostSnmp,Test-WSMan,Trace-Command,Unblock-File,Unblock-Tpm,Undo-DtiagnosticTransaction,Undo-Transaction,Uninstall-AIPScanner,Uninstall-MIPNetworkDiscovery,Uninstall-Package,Uninstall-ProvisioningPackage,Uninstall-TrustedProvisioningCertificate,Unprotect-CmsMessage,Unpublish-AppvClientPackage,Unregister-Event,Unregister-PackageSource,Unregister-PSSessionConfiguration,Unregister-ScheduledJob,Unregister-UevTemplate,Unregister-WindowsDeveloperLicense,Update-AIPScanner,Update-FormatData,Update-Help,Update-List,Update-Tools,Update-TypeData,Update-UevTemplate,Update-WIMBootEntry,Use-PowerCLIContext,Use-Transaction,Use-WindowsUnattend,Wait-Debugger,Wait-Event,Wait-Job,Wait-Process,Wait-Task,Wait-Tools,Where-Object,Write-Debug,Write-Error,Write-EventLog,Write-Host,Write-Information,Write-Output,Write-Progress,Write-Verbose,Write-Warning
	},
	morekeywords={
		Get-Testfunktion,Add-BataCacheExtension,Add-BitLockerKeyProtector,Add-ConditionalFormatting,Add-DnsClientNrptRule,Add-DtcClusterTMMapping,Add-EtwTraceProvider,Add-ExcelChart,Add-ExcelDataValidationRule,Add-ExcelName,Add-ExcelTable,Add-InitiatorIdToMaskingSet,Add-MpPreference,Add-NetEventNetworkAdapter,Add-NetEventPacketCaptureProvider,Add-NetEventProvider,Add-NetEventVFPProvider,Add-NetEventVmNetworkAdapter,Add-NetEventVmSwitch,Add-NetEventVmSwitchProvider,Add-NetEventWFPCaptureProvider,Add-NetIPHttpsCertBinding,Add-NetLbfoTeamMember,Add-NetLbfoTeamNic,Add-NetNatExternalAddress,Add-NetNatStaticMapping,Add-NetSwitchTeamMember,Add-Odbsn,Add-PartitionAccessPath,Add-PhysicalDisk,Add-PivotTable,Add-Printer,Add-PrinterDriver,Add-PrinterPort,Add-StorageFaultDomain,Add-TargetPortToMaskingSet,Add-VirtualDiskToMaskingSet,Add-VpnConnection,Add-VpnConnectionRoute,Add-VpnConnectionTriggerApplication,Add-VpnConnectionTriggerDnsConfiguration,Add-VpnConnectionTriggerTrustedNetwork,Add-Worksheet,AfterAll,AfterEach,Assert-MockCalled,Assert-VerifiableMocks,Backup-BitLockerKeyProtector,BackupToAAD-BitLockerKeyProtector,BarChart,BeforeAll,BeforeEach,Block-FileShareAccess,Block-SmbShareAccess,Clear-AssignedAccess,Clear-BCCache,Clear-BitLockerAutoUnlock,Clear-Disk,Clear-DnsClientCache,Clear-FileStorageTier,Clear-Host,Clear-Menu,Clear-PcsvDeviceLog,Clear-StorageBusDisk,Clear-StorageDiagnosticInfo,Close-ExcelPackage,Close-SmbOpenFile,Close-SmbSession,ColumnChart,Compare-Worksheet,Compress-Archive,Configuration,Connect-IscsiTarget,Connect-VirtualDisk,Context,convert,Convert-ExcelRangeToImage,ConvertFrom-ExcelData,ConvertFrom-ExcelSheet,ConvertFrom-ExcelToSQLInsert,ConvertFrom-SddlString,ConvertTo-ExcelXlsx,Copy-ExcelWorksheet,Copy-NetFirewallRule,Copy-NetIPsecMainModeCryptoSet,Copy-NetIPsecMainModeRule,Copy-NetIPsecPhase1AuthSet,Copy-NetIPsecPhase2AuthSet,Copy-NetIPsecQuickModeCryptoSet,Copy-NetIPsecRule,Debug-FileShare,Debug-MMAppPrelaunch,Debug-StorageSubSystem,Debug-Volume,Describe,Disable-BC,Disable-Bowngrading,Disable-BCServeOnBattery,Disable-BitLocker,Disable-BitLockerAutoUnlock,Disable-DAManualEntryPointSelection,Disable-Dsebug,Disable-MMAgent,Disable-NetAdapter,Disable-NetAdapterBinding,Disable-NetAdapterChecksumOffload,Disable-NetAdapterEncapsulatedPacketTaskOffload,Disable-NetAdapterIPsecOffload,Disable-NetAdapterLso,Disable-NetAdapterPacketDirect,Disable-NetAdapterPowerManagement,Disable-NetAdapterQos,Disable-NetAdapterRdma,Disable-NetAdapterRsc,Disable-NetAdapterRss,Disable-NetAdapterSriov,Disable-NetAdapterUso,Disable-NetAdapterVmq,Disable-NetDnsTransitionConfiguration,Disable-NetFirewallRule,Disable-NetIPHttpsProfile,Disable-NetIPsecMainModeRule,Disable-NetIPsecRule,Disable-NetNatTransitionConfiguration,Disable-NetworkSwitchEthernetPort,Disable-NetworkSwitchFeature,Disable-NetworkSwitchVlan,Disable-OdbcPerfCounter,Disable-PhysicalDiskIdentification,Disable-PnpDevice,Disable-PSTrace,Disable-PSWSManCombinedTrace,Disable-ScheduledTask,Disable-SmbDelegation,Disable-StorageBusCache,Disable-StorageBusDisk,Disable-StorageEnclosureIdentification,Disable-StorageEnclosurePower,Disable-StorageHighAvailability,Disable-StorageMaintenanceMode,Disable-WdacBidTrace,Disable-WSManTrace,Disconnect-IscsiTarget,Disconnect-VirtualDisk,Dismount-DiskImage,DoChart,Enable-Bistributed,Enable-Bowngrading,Enable-BCHostedClient,Enable-BCHostedServer,Enable-BCLocal,Enable-BCServeOnBattery,Enable-BitLocker,Enable-BitLockerAutoUnlock,Enable-DAManualEntryPointSelection,Enable-Dsebug,Enable-MMAgent,Enable-NetAdapter,Enable-NetAdapterBinding,Enable-NetAdapterChecksumOffload,Enable-NetAdapterEncapsulatedPacketTaskOffload,Enable-NetAdapterIPsecOffload,Enable-NetAdapterLso,Enable-NetAdapterPacketDirect,Enable-NetAdapterPowerManagement,Enable-NetAdapterQos,Enable-NetAdapterRdma,Enable-NetAdapterRsc,Enable-NetAdapterRss,Enable-NetAdapterSriov,Enable-NetAdapterUso,Enable-NetAdapterVmq,Enable-NetDnsTransitionConfiguration,Enable-NetFirewallRule,Enable-NetIPHttpsProfile,Enable-NetIPsecMainModeRule,Enable-NetIPsecRule,Enable-NetNatTransitionConfiguration,Enable-NetworkSwitchEthernetPort,Enable-NetworkSwitchFeature,Enable-NetworkSwitchVlan,Enable-OdbcPerfCounter,EnableParameterCompleters,Enable-PhysicalDiskIdentification,Enable-PnpDevice,Enable-PSTrace,Enable-PSWSManCombinedTrace,Enable-ScheduledTask,Enable-SmbDelegation,Enable-StorageBusCache,Enable-StorageBusDisk,Enable-StorageEnclosureIdentification,Enable-StorageEnclosurePower,Enable-StorageHighAvailability,Enable-StorageMaintenanceMode,Enable-WdacBidTrace,Enable-WSManTrace,Expand-Archive,Expand-NumberFormat,Export-BCCachePackage,Export-BCSecretKey,Export-Excel,Export-ODataEndpointProxy,Export-ScheduledTask,Find-Command,Find-DscResource,Find-Module,Find-NetIPsecRule,Find-NetRoute,Find-RoleCapability,Find-Script,Flush-EtwTraceSession,Format-Hex,Format-Volume,Get-AppBackgroundTask,Get-AppvVirtualProcess,Get-AppxLastError,Get-AppxLog,Get-AssignedAccess,Get-AutologgerConfig,Get-BCClientConfiguration,Get-BCContentServerConfiguration,Get-BataCache,Get-BataCacheExtension,Get-BCHashCache,Get-BCHostedCacheServerConfiguration,Get-BCNetworkConfiguration,Get-BCStatus,Get-BitLockerVolume,Get-CisCommand,Get-ClusteredScheduledTask,Get-DAClientExperienceConfiguration,Get-DAConnectionStatus,Get-DAEntryPointTableItem,Get-DedupProperties,Get-Disk,Get-DiskImage,Get-DiskStorageNodeView,Get-DnsClient,Get-DnsClientCache,Get-DnsClientGlobalSetting,Get-DnsClientNrptGlobal,Get-DnsClientNrptPolicy,Get-DnsClientNrptRule,Get-DnsClientServerAddress,Get-DscConfiguration,Get-DscConfigurationStatus,Get-DscLocalConfigurationManager,Get-DscResource,Get-Dtc,Get-DtcAdvancedHostSetting,Get-DtcAdvancedSetting,Get-DtcClusterDefault,Get-DtcClusterTMMapping,Get-Dtefault,Get-DtcLog,Get-DtcNetworkSetting,Get-DtcTransaction,Get-DtcTransactionsStatistics,Get-DtcTransactionsTraceSession,Get-DtcTransactionsTraceSetting,Get-EtwTraceProvider,Get-EtwTraceSession,Get-ExcelColumnName,Get-ExcelFileSummary,Get-ExcelSheetInfo,Get-ExcelWorkbookInfo,Get-FileHash,Get-FileIntegrity,Get-FileShare,Get-FileShareAccessControlEntry,Get-FileStorageTier,Get-HtmlTable,Get-InitiatorId,Get-InitiatorPort,Get-InstalledModule,Get-InstalledScript,Get-InstallPath,Get-IscsiConnection,Get-IscsiSession,Get-IscsiTarget,Get-IscsiTargetPortal,Get-IseSnippet,Get-LogProperties,Get-MaskingSet,Get-MMAgent,Get-MockDynamicParameters,Get-MpComputerStatus,Get-MpPreference,Get-MpThreat,Get-MpThreatCatalog,Get-MpThreatDetection,Get-NCSIPolicyConfiguration,Get-Net6to4Configuration,Get-NetAdapter,Get-NetAdapterAdvancedProperty,Get-NetAdapterBinding,Get-NetAdapterChecksumOffload,Get-NetAdapterEncapsulatedPacketTaskOffload,Get-NetAdapterHardwareInfo,Get-NetAdapterIPsecOffload,Get-NetAdapterLso,Get-NetAdapterPacketDirect,Get-NetAdapterPowerManagement,Get-NetAdapterQos,Get-NetAdapterRdma,Get-NetAdapterRsc,Get-NetAdapterRss,Get-NetAdapterSriov,Get-NetAdapterSriovVf,Get-NetAdapterStatistics,Get-NetAdapterUso,Get-NetAdapterVmq,Get-NetAdapterVMQQueue,Get-NetAdapterVPort,Get-NetCompartment,Get-NetConnectionProfile,Get-NetDnsTransitionConfiguration,Get-NetDnsTransitionMonitoring,Get-NetEventNetworkAdapter,Get-NetEventPacketCaptureProvider,Get-NetEventProvider,Get-NetEventSession,Get-NetEventVFPProvider,Get-NetEventVmNetworkAdapter,Get-NetEventVmSwitch,Get-NetEventVmSwitchProvider,Get-NetEventWFPCaptureProvider,Get-NetFirewallAddressFilter,Get-NetFirewallApplicationFilter,Get-NetFirewallInterfaceFilter,Get-NetFirewallInterfaceTypeFilter,Get-NetFirewallPortFilter,Get-NetFirewallProfile,Get-NetFirewallRule,Get-NetFirewallSecurityFilter,Get-NetFirewallServiceFilter,Get-NetFirewallSetting,Get-NetIPAddress,Get-NetIPConfiguration,Get-NetIPHttpsConfiguration,Get-NetIPHttpsState,Get-NetIPInterface,Get-NetIPseospSetting,Get-NetIPsecMainModeCryptoSet,Get-NetIPsecMainModeRule,Get-NetIPsecMainModeSA,Get-NetIPsecPhase1AuthSet,Get-NetIPsecPhase2AuthSet,Get-NetIPsecQuickModeCryptoSet,Get-NetIPsecQuickModeSA,Get-NetIPsecRule,Get-NetIPv4Protocol,Get-NetIPv6Protocol,Get-NetIsatapConfiguration,Get-NetLbfoTeam,Get-NetLbfoTeamMember,Get-NetLbfoTeamNic,Get-NetNat,Get-NetNatExternalAddress,Get-NetNatGlobal,Get-NetNatSession,Get-NetNatStaticMapping,Get-NetNatTransitionConfiguration,Get-NetNatTransitionMonitoring,Get-NetNeighbor,Get-NetOffloadGlobalSetting,Get-NetPrefixPolicy,Get-NetQosPolicy,Get-NetRoute,Get-NetSwitchTeam,Get-NetSwitchTeamMember,Get-NetTCPConnection,Get-NetTCPSetting,Get-NetTeredoConfiguration,Get-NetTeredoState,Get-NetTransportFilter,Get-NetUDPEndpoint,Get-NetUDPSetting,Get-NetworkSwitchEthernetPort,Get-NetworkSwitchFeature,Get-NetworkSwitchGlobalData,Get-NetworkSwitchVlan,Get-Odbriver,Get-Odbsn,Get-OdbcPerfCounter,Get-OffloadDataTransferSetting,Get-OperationValidation,Get-Partition,Get-PartitionSupportedSize,Get-PcsvDevice,Get-PcsvDeviceLog,Get-PhysicalDisk,Get-PhysicalDiskStorageNodeView,Get-PhysicalExtent,Get-PhysicalExtentAssociation,Get-PnpDevice,Get-PnpDeviceProperty,Get-PowerCLICommunity,Get-PowerCLIHelp,Get-PrintConfiguration,Get-Printer,Get-PrinterDriver,Get-PrinterPort,Get-PrinterProperty,Get-PrintJob,Get-ProcessMonitor,Get-PSRepository,Get-PSVersion,Get-Range,Get-ResiliencySetting,Get-ScheduledTask,Get-ScheduledTaskInfo,Get-SmbBandWidthLimit,Get-SmbClientConfiguration,Get-SmbClientNetworkInterface,Get-SmbConnection,Get-SmbDelegation,Get-SmbGlobalMapping,Get-SmbMapping,Get-SmbMultichannelConnection,Get-SmbMultichannelConstraint,Get-SmbOpenFile,Get-SmbServerConfiguration,Get-SmbServerNetworkInterface,Get-SmbSession,Get-SmbShare,Get-SmbShareAccess,Get-SmbWitnessClient,Get-StartApps,Get-StorageAdvancedProperty,Get-StorageBusBinding,Get-StorageBusDisk,Get-StorageChassis,Get-StorageDiagnosticInfo,Get-StorageEnclosure,Get-StorageEnclosureStorageNodeView,Get-StorageEnclosureVendorData,Get-StorageExtendedStatus,Get-StorageFaultDomain,Get-StorageFileServer,Get-StorageFirmwareInformation,Get-StorageHealthAction,Get-StorageHealthReport,Get-StorageHealthSetting,Get-StorageHistory,Get-StorageJob,Get-StorageNode,Get-StoragePool,Get-StorageProvider,Get-StorageRack,Get-StorageReliabilityCounter,Get-StorageScaleUnit,Get-StorageSetting,Get-StorageSite,Get-StorageSubSystem,Get-StorageTier,Get-StorageTierSupportedSize,Get-SupportedClusterSizes,Get-SupportedFileSystems,Get-SystemInfo,Get-SystemInfo-Hinweise,Get-TargetPort,Get-TargetPortal,Get-TestDriveItem,Get-Verb,Get-VICommand,Get-VirtualDisk,Get-VirtualDiskSupportedSize,Get-Volume,Get-VolumeCorruptionCount,Get-VolumeScrubPolicy,Get-VpnConnection,Get-VpnConnectionTrigger,Get-WdacBidTrace,Get-WindowsUpdateLog,Get-Windows-Updates-Approve,Get-Windows-Updates-Hinweise,Get-WUAVersion,Get-WUIsPendingReboot,Get-WULastInstallationDate,Get-WULastScanSuccessDate,Get-XYRange,Grant-FileShareAccess,Grant-SmbShareAccess,Handle-Menu,Handle-Menu-Test,help,Hide-VirtualDisk,HookGetViewAutoCompleter,Import-BCCachePackage,Import-BCSecretKey,Import-Excel,Import-Html,Import-IseSnippet,Import-PowerShellDataFile,ImportSystemModules,Import-UPS,Import-USPS,In,Initialize-Disk,InModuleScope,Install-Dtc,Install-Module,Install-Script,Install-WUUpdates,Invoke-AsWorkflow,Invoke-Mock,Invoke-OperationValidation,Invoke-Pester,Invoke-Sum,It,Join-Worksheet,Limit-Var,LineChart,Lock-BitLocker,Merge-MultipleSheets,Merge-Worksheet,mkdir,Mock,more,Mount-DiskImage,Move-SmbWitnessClient,New-AutologgerConfig,New-ConditionalFormattingIconSet,New-ConditionalText,New-DAEntryPointTableItem,New-DatastoreDrive,New-DscChecksum,New-EapConfiguration,New-EtwTraceSession,New-ExcelChartDefinition,New-ExcelStyle,New-FileShare,New-Fixture,New-Guid,New-IscsiTargetPortal,New-IseSnippet,New-MaskingSet,New-NetAdapterAdvancedProperty,New-NetEventSession,New-NetFirewallRule,New-NetIPAddress,New-NetIPHttpsConfiguration,New-NetIPseospSetting,New-NetIPsecMainModeCryptoSet,New-NetIPsecMainModeRule,New-NetIPsecPhase1AuthSet,New-NetIPsecPhase2AuthSet,New-NetIPsecQuickModeCryptoSet,New-NetIPsecRule,New-NetLbfoTeam,New-NetNat,New-NetNatTransitionConfiguration,New-NetNeighbor,New-NetQosPolicy,New-NetRoute,New-NetSwitchTeam,New-NetTransportFilter,New-NetworkSwitchVlan,New-Partition,New-PesterOption,New-PivotTableDefinition,New-Plot,New-PSItem,New-PSWorkflowSession,New-ScheduledTask,New-ScheduledTaskAction,New-ScheduledTaskPrincipal,New-ScheduledTaskSettingsSet,New-ScheduledTaskTrigger,New-ScriptFileInfo,New-SmbGlobalMapping,New-SmbMapping,New-SmbMultichannelConstraint,New-SmbShare,New-StorageBusBinding,New-StorageBusCacheStore,New-StorageFileServer,New-StoragePool,New-StorageSubsystemVirtualDisk,New-StorageTier,New-TemporaryFile,New-VIInventoryDrive,New-VirtualDisk,New-VirtualDiskClone,New-VirtualDiskSnapshot,New-Volume,New-VpnServerAddress,Open-ExcelPackage,Open-NetGPO,Optimize-StoragePool,Optimize-Volume,oss,Pause,PieChart,Pivot,prompt,PSConsoleHostReadLine,Publish-BCFileContent,Publish-BCWebContent,Publish-Module,Publish-Script,Read-PrinterNfcTag,Register-ClusteredScheduledTask,Register-DnsClient,Register-IscsiSession,Register-PSRepository,Register-ScheduledTask,Register-StorageSubsystem,Remove-AutologgerConfig,Remove-BataCacheExtension,Remove-BitLockerKeyProtector,Remove-DAEntryPointTableItem,Remove-DnsClientNrptRule,Remove-DscConfigurationDocument,Remove-DtcClusterTMMapping,Remove-EtwTraceProvider,Remove-FileShare,Remove-InitiatorId,Remove-InitiatorIdFromMaskingSet,Remove-IscsiTargetPortal,Remove-MaskingSet,Remove-MpPreference,Remove-MpThreat,Remove-NetAdapterAdvancedProperty,Remove-NetEventNetworkAdapter,Remove-NetEventPacketCaptureProvider,Remove-NetEventProvider,Remove-NetEventSession,Remove-NetEventVFPProvider,Remove-NetEventVmNetworkAdapter,Remove-NetEventVmSwitch,Remove-NetEventVmSwitchProvider,Remove-NetEventWFPCaptureProvider,Remove-NetFirewallRule,Remove-NetIPAddress,Remove-NetIPHttpsCertBinding,Remove-NetIPHttpsConfiguration,Remove-NetIPseospSetting,Remove-NetIPsecMainModeCryptoSet,Remove-NetIPsecMainModeRule,Remove-NetIPsecMainModeSA,Remove-NetIPsecPhase1AuthSet,Remove-NetIPsecPhase2AuthSet,Remove-NetIPsecQuickModeCryptoSet,Remove-NetIPsecQuickModeSA,Remove-NetIPsecRule,Remove-NetLbfoTeam,Remove-NetLbfoTeamMember,Remove-NetLbfoTeamNic,Remove-NetNat,Remove-NetNatExternalAddress,Remove-NetNatStaticMapping,Remove-NetNatTransitionConfiguration,Remove-NetNeighbor,Remove-NetQosPolicy,Remove-NetRoute,Remove-NetSwitchTeam,Remove-NetSwitchTeamMember,Remove-NetTransportFilter,Remove-NetworkSwitchEthernetPortIPAddress,Remove-NetworkSwitchVlan,Remove-Odbsn,Remove-Partition,Remove-PartitionAccessPath,Remove-PhysicalDisk,Remove-Printer,Remove-PrinterDriver,Remove-PrinterPort,Remove-PrintJob,Remove-SmbBandwidthLimit,Remove-SMBComponent,Remove-SmbGlobalMapping,Remove-SmbMapping,Remove-SmbMultichannelConstraint,Remove-SmbShare,Remove-StorageBusBinding,Remove-StorageFaultDomain,Remove-StorageFileServer,Remove-StorageHealthIntent,Remove-StorageHealthSetting,Remove-StoragePool,Remove-StorageTier,Remove-TargetPortFromMaskingSet,Remove-VirtualDisk,Remove-VirtualDiskFromMaskingSet,Remove-VpnConnection,Remove-VpnConnectionRoute,Remove-VpnConnectionTriggerApplication,Remove-VpnConnectionTriggerDnsConfiguration,Remove-VpnConnectionTriggerTrustedNetwork,Remove-Worksheet,Rename-DAEntryPointTableItem,Rename-MaskingSet,Rename-NetAdapter,Rename-NetFirewallRule,Rename-NetIPHttpsConfiguration,Rename-NetIPsecMainModeCryptoSet,Rename-NetIPsecMainModeRule,Rename-NetIPsecPhase1AuthSet,Rename-NetIPsecPhase2AuthSet,Rename-NetIPsecQuickModeCryptoSet,Rename-NetIPsecRule,Rename-NetLbfoTeam,Rename-NetSwitchTeam,Rename-Printer,Repair-FileIntegrity,Repair-VirtualDisk,Repair-Volume,Reset-BC,Reset-DAClientExperienceConfiguration,Reset-DAEntryPointTableItem,Reset-DtcLog,Reset-NCSIPolicyConfiguration,Reset-Net6to4Configuration,Reset-NetAdapterAdvancedProperty,Reset-NetDnsTransitionConfiguration,Reset-NetIPHttpsConfiguration,Reset-NetIsatapConfiguration,Reset-NetTeredoConfiguration,Reset-PhysicalDisk,Reset-StorageReliabilityCounter,Resize-Partition,Resize-StorageTier,Resize-VirtualDisk,Restart-NetAdapter,Restart-PcsvDevice,Restart-PrintJob,Restore-DscConfiguration,Restore-NetworkSwitchConfiguration,Resume-BitLocker,Resume-PrintJob,Resume-StorageBusDisk,Revoke-FileShareAccess,Revoke-SmbShareAccess,SafeGetCommand,Save-EtwTraceSession,Save-Module,Save-NetGPO,Save-NetworkSwitchConfiguration,Save-Script,Select-Worksheet,Send-EtwTraceSession,Send-SQLDataToExcel,Set-AssignedAccess,Set-BCAuthentication,Set-BCCache,Set-BataCacheEntryMaxAge,Set-BCMinSMBLatency,Set-BCSecretKey,Set-CellStyle,Set-ClusteredScheduledTask,Set-DAClientExperienceConfiguration,Set-DAEntryPointTableItem,Set-Disk,Set-DnsClient,Set-DnsClientGlobalSetting,Set-DnsClientNrptGlobal,Set-DnsClientNrptRule,Set-DnsClientServerAddress,Set-DtcAdvancedHostSetting,Set-DtcAdvancedSetting,Set-DtcClusterDefault,Set-DtcClusterTMMapping,Set-Dtefault,Set-DtcLog,Set-DtcNetworkSetting,Set-DtcTransaction,Set-DtcTransactionsTraceSession,Set-DtcTransactionsTraceSetting,Set-DynamicParameterVariables,Set-EtwTraceProvider,Set-ExcelColumn,Set-ExcelRange,Set-ExcelRow,Set-FileIntegrity,Set-FileShare,Set-FileStorageTier,Set-InitiatorPort,Set-IscsiChapSecret,Set-LogProperties,Set-MMAgent,Set-MpPreference,Set-NCSIPolicyConfiguration,Set-Net6to4Configuration,Set-NetAdapter,Set-NetAdapterAdvancedProperty,Set-NetAdapterBinding,Set-NetAdapterChecksumOffload,Set-NetAdapterEncapsulatedPacketTaskOffload,Set-NetAdapterIPsecOffload,Set-NetAdapterLso,Set-NetAdapterPacketDirect,Set-NetAdapterPowerManagement,Set-NetAdapterQos,Set-NetAdapterRdma,Set-NetAdapterRsc,Set-NetAdapterRss,Set-NetAdapterSriov,Set-NetAdapterUso,Set-NetAdapterVmq,Set-NetConnectionProfile,Set-NetDnsTransitionConfiguration,Set-NetEventPacketCaptureProvider,Set-NetEventProvider,Set-NetEventSession,Set-NetEventVFPProvider,Set-NetEventVmSwitchProvider,Set-NetEventWFPCaptureProvider,Set-NetFirewallAddressFilter,Set-NetFirewallApplicationFilter,Set-NetFirewallInterfaceFilter,Set-NetFirewallInterfaceTypeFilter,Set-NetFirewallPortFilter,Set-NetFirewallProfile,Set-NetFirewallRule,Set-NetFirewallSecurityFilter,Set-NetFirewallServiceFilter,Set-NetFirewallSetting,Set-NetIPAddress,Set-NetIPHttpsConfiguration,Set-NetIPInterface,Set-NetIPseospSetting,Set-NetIPsecMainModeCryptoSet,Set-NetIPsecMainModeRule,Set-NetIPsecPhase1AuthSet,Set-NetIPsecPhase2AuthSet,Set-NetIPsecQuickModeCryptoSet,Set-NetIPsecRule,Set-NetIPv4Protocol,Set-NetIPv6Protocol,Set-NetIsatapConfiguration,Set-NetLbfoTeam,Set-NetLbfoTeamMember,Set-NetLbfoTeamNic,Set-NetNat,Set-NetNatGlobal,Set-NetNatTransitionConfiguration,Set-NetNeighbor,Set-NetOffloadGlobalSetting,Set-NetQosPolicy,Set-NetRoute,Set-NetTCPSetting,Set-NetTeredoConfiguration,Set-NetUDPSetting,Set-NetworkSwitchEthernetPortIPAddress,Set-NetworkSwitchPortMode,Set-NetworkSwitchPortProperty,Set-NetworkSwitchVlanProperty,Set-Odbriver,Set-Odbsn,Set-Partition,Set-PcsvDeviceBootConfiguration,Set-PcsvDeviceNetworkConfiguration,Set-PcsvDeviceUserPassword,Set-PhysicalDisk,Set-PrintConfiguration,Set-Printer,Set-PrinterProperty,Set-PSRepository,Set-ResiliencySetting,Set-ScheduledTask,Set-SmbBandwidthLimit,Set-SmbClientConfiguration,Set-SmbPathAcl,Set-SmbServerConfiguration,Set-SmbShare,Set-StorageBusProfile,Set-StorageFileServer,Set-StorageHealthSetting,Set-StoragePool,Set-StorageProvider,Set-StorageSetting,Set-StorageSubSystem,Set-StorageTier,Set-TestInconclusive,Setup,Set-VirtualDisk,Set-Volume,Set-VolumeScrubPolicy,Set-VpnConnection,Set-VpnConnectionIPsecConfiguration,Set-VpnConnectionProxy,Set-VpnConnectionTriggerDnsConfiguration,Set-VpnConnectionTriggerTrustedNetwork,Set-WorksheetProtection,Should,Show-Menu,Show-NetFirewallRule,Show-NetIPsecRule,Show-StorageHistory,Show-VirtualDisk,Start-AppBackgroundTask,Start-AppvVirtualProcess,Start-AutologgerConfig,Start-Dtc,Start-DtcTransactionsTraceSession,Start-EtwTraceSession,Start-MpScan,Start-MpWDOScan,Start-NetEventSession,Start-PcsvDevice,Start-ScheduledTask,Start-StorageDiagnosticLog,Start-Trace,Start-WUScan,Stop-DscConfiguration,Stop-Dtc,Stop-DtcTransactionsTraceSession,Stop-EtwTraceSession,Stop-NetEventSession,Stop-PcsvDevice,Stop-ScheduledTask,Stop-StorageDiagnosticLog,Stop-StorageJob,Stop-Trace,Suspend-BitLocker,Suspend-PrintJob,Suspend-StorageBusDisk,Sync-NetIPsecRule,TabExpansion2,TabExpansionDefault,Test-Boolean,Test-Date,Test-Dtc,Test-Integer,Test-NetConnection,Test-Number,Test-ScriptFileInfo,Test-String,Unblock-FileShareAccess,Unblock-SmbShareAccess,Uninstall-Dtc,Uninstall-Module,Uninstall-Script,Unlock-BitLocker,Unregister-AppBackgroundTask,Unregister-ClusteredScheduledTask,Unregister-IscsiSession,Unregister-PSRepository,Unregister-ScheduledTask,Unregister-StorageSubsystem,Update-AutologgerConfig,Update-Disk,Update-DscConfiguration,Update-EtwTraceSession,Update-FirstObjectProperties,Update-HostStorageCache,Update-IscsiTarget,Update-IscsiTargetPortal,Update-Module,Update-ModuleManifest,Update-MpSignature,Update-NetIPsecRule,Update-Script,Update-ScriptFileInfo,Update-SmbMultichannelConnection,Update-StorageFirmware,Update-StoragePool,Update-StorageProviderCache,Write-DtcTransactionsTraceSession,Write-PrinterNfcTag,Write-VolumeCache
	},
	morekeywords={Do,Else,For,ForEach,Function,If,In,Until,While},
	alsodigit={-},
	sensitive=false,
	morecomment=[l]{\#},
	morecomment=[n]{<\#}{\#>},
	morestring=[b]{"},
	morestring=[b]{'},
	morestring=[s]{@'}{'@},
	morestring=[s]{@"}{"@}
}

\lstset{
  aboveskip=0.6\baselineskip,
  commentstyle=\color{lst-green},
  basicstyle=\small\ttfamily,
  backgroundcolor=\color{lst-gray},
  breaklines=true,
  captionpos=b,
  columns=fixed,
  extendedchars=true,
  frame=single,
  framesep=2pt,
  keepspaces=true,
  keywordstyle=\color{lst-blue},
  language={PowerShell},
  numbers=left,
  numberstyle=\tiny\ttfamily,
  showstringspaces=false,
  stringstyle=\color{lst-red},
  tabsize=2,
  %stepnumber=0
  stepnumber=1,
}


\usepackage[
  open,
  openlevel=2,
  atend,
  numbered
]{bookmark}

%\addtokomafont{headsepline}{\color{gray}}
%\addtokomafont{footsepline}{\color{gray}}
%\renewcommand*{\chaptermarkformat}{}
%\automark[]{chapter}

%\ihead*{\color{\chaptercolor}Abschnitt \thesection: ~ \rigmarkmark }
%\ohead{\color{\chaptercolor}\headmark}


%\ohead*{\color{\chaptercolor}Kapitel \thechapter:~\headmark }
%\cfoot*[\color{\chaptercolor}\pagemark]{\color{\chaptercolor}\pagemark}
%\ifoot*[\color{\chaptercolor}HTWK Leipzig]{\color{\chaptercolor}HTWK Leipzig}
%\ofoot*[\color{\chaptercolor}Maurice Götze]{\color{\chaptercolor}Maurice Götze}


%\fancypagestyle{plain}{% Redefine ``plain'' style for chapter boundaries
%    \fancyhf{} % clear all header and footer fields  
%    \fancyfoot[C]{\color{\chaptercolor}\thepage} % except the center 
%%    %\fancyfoot[R]{\color{\chaptercolor}Maurice Götze}
 %   %\fancyfoot[L]{\color{\chaptercolor}HTWK Leipzig}
 %   \renewcommand{\headrulewidth}{0pt}  
 %   \renewcommand{\footrulewidth}{0.5pt}
%}


